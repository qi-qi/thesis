  %%%%%%%%%%%%%%%%%%%%%%%%%%%%%%%%%%%%%%% -*- coding: utf-8; mode: latex -*- %%
  %
%%%%%                       CHAPTER
 %%%
  %

% $Id: 7200-magna-aliqua.tex,v 1.1 2007/11/23 09:52:46 david Exp $
% $Log: 7200-magna-aliqua.tex,v $
% Revision 1.1  2007/11/23 09:52:46  david
% *** empty log message ***
%
%

  %%%%%%%%%%%%%%%%%%%%%%%%%%%%%%%%%%%%%%%%%%%%%%%%%%%%%%%%%%%%%%%%%%%%%%%%%%%%%
  %
%%%%%                    HEAD MATTER
 %%%
  %

\chapter{Conclusion and Future Work}
%\addcontentsline{lof}{chapter}{\thechapter\quad Nihil Molestiae}
%\addcontentsline{lot}{chapter}{\thechapter\quad Nihil Molestiae}
\label{ch:conclusion}

%\begin{quotation}
%  {\small\it Neque porro quisquam est qui dolorem ipsum quia dolor sit amet, consectetur, adipisci velit...}
%
%{\small\it -- Cerico}
%\end{quotation}



  %%%%%%%%%%%%%%%%%%%%%%%%%%%%%%%%%%%%%%%%%%%%%%%%%%%%%%%%%%%%%%%%%%%%%%%%%%%%%
  %
%%%%%                        FIRST SECTION
 %%%
  %

\section{Conclusion}

In this thesis, we provide a solution for Hop-HDFS based on optimistic concurrency control with snapshot isolation on semantic related group to improve the operation throughput while maintaining the strong consistency semantics in HDFS.

\noindent First, we discuss the architectures of related distributed file systems, including Google File System, HDFS and Hop-HDFS. With focus on their namespace concurrency control schemes, we analyzes the limitation of HDFS's NameNode implementation and provide an overview of Hop-HDFS illustrating how we overcome those problems in the distributed NameNode architecture.

\noindent MySQL Cluster is selected to be the distributed in-memory storage layer for the metadata in Hop-HDFS due to its the high operation throughput and high reliability. However, the trade off is that the NDB cluster storage engine  of MySQL cluster supports only the \textit{READ COMMITTED} transaction isolation level. \textit{Anomalies} like fuzzy read, phantom, write skew will appear because the write results in transactions will be exposed to reads in different concurrent transactions without proper implementation.

\noindent Then, based on optimistic concurrency control with snapshot isolation on semantic related group, we demonstrate how concurrency is improved and anomalies - \textit{fuzzy read, phantom, write skew} are precluded, so that the strong consistency semantics in HDFS is maintained.

\noindent Finally, as a proof of concept, we implemented the OCC version for the operation \textit{mkdirs} and also give a detailed evaluation on it compared with the PCC version. Our solution outperforms previous work of Hop-HDFS up to 70 \%. Under heavy workload, the single NameNode performance of HDFS is just a slightly better than OCC. We believe that OCC can greatly outperform HDFS in Hop-HDFS multiple NameNodes architecture. The correctness of our implementation has been validated by 300+ Apache HDFS unit tests passing.

  %%%%%%%%%%%%%%%%%%%%%%%%%%%%%%%%%%%%%%%%%%%%%%%%%%%%%%%%%%%%%%%%%%%%%%%%%%%%%
  %
%%%%%                      SECOND SECTION
 %%%
  %

\section{Future Work}

The result of our OCC solution is promising. Other operations in Hop-HDFS can also adopt the same algorithm to achieve better performance.

\noindent Future evaluation on Hop-HDFS in multiple NameNodes architecture with OCC solution is needed to prove that it can achieve better performance than HDFS in single NameNode architecture.

