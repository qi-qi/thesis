  %%%%%%%%%%%%%%%%%%%%%%%%%%%%%%%%%%%%%%% -*- coding: utf-8; mode: latex -*- %%
  %
%%%%%                       CHAPTER
 %%%
  %

% $Id: 3100-nihil-molestiae.tex,v 1.1 2007/11/23 09:52:43 david Exp $
% $Log: 3100-nihil-molestiae.tex,v $
% Revision 1.1  2007/11/23 09:52:43  david
% *** empty log message ***
%
%

  %%%%%%%%%%%%%%%%%%%%%%%%%%%%%%%%%%%%%%%%%%%%%%%%%%%%%%%%%%%%%%%%%%%%%%%%%%%%%
  %
%%%%%                    HEAD MATTER
 %%%
  %

\chapter{Design and Implementation}
%\addcontentsline{lof}{chapter}{\thechapter\quad Nihil Molestiae}
%\addcontentsline{lot}{chapter}{\thechapter\quad Nihil Molestiae}
\label{ch:Design}

\section{Resolving the Semantic Related Group}

Resolving the semantic related group for each transaction is the fundamental step to preclude \textit{anomalies} in our implementation. Because it defines the constraints between individual data rows. In Hop-HDFS, each metadata operation is implemented as an individual transaction running by a worker thread. Any metadata operation related to the namespace will have one or two input parameters, called \textit{Path}. Here's two examples for methods in the Filesystem API:

\begin{itemize}[noitemsep]
	\item boolean \textbf{mkdirs} (Path \textit{f}): \textit{f} is the path of the INodeDirectory to be created
	\item boolean \textbf{rename} (Path \textit{src}, Path \textit{dst}): \textit{src} is the path to be renamed, \textit{dst} is the new path after rename
\end{itemize} 

\noindent Each \textit{Path} object is related to a string representation of the "/" based absolute path name. For example, in Figure~\ref{fig:hoptree}, the path for INode \textit{h} is: 
\begin{center}
	/a/d/f/h
\end{center}

\noindent Therefore, with the preservation of the \textit{directed tree structure}, we can resolving a semantic related group for each INode along the edge of ancestors as a \textit{LinkedList}. Therefore, the semantic related group representation for INode \textit{h} is:
\begin{center}
	h: \{/-$>$a-$>$d-$>$f\}
\end{center}

\noindent In other words, for each row in \textit{inodes table}, we can figure out its semantic related rows as shown in Table~\ref{table:semanticrelatedTable}.

\begin{table}[h]
	\centering
	\begin{tabular}{|c|c|c|c|c|}
		\hline
		~ & \textbf{id} & \textbf{parent\_id} & \textbf{name} & \textbf{other parameters...} \\ \hline
		Related * & 1 & 0 & / & ... \\ \hline
		Related * & 2 & 1 & a & ... \\ \hline
		~ & 3 & 1 & b & ... \\ \hline
		~ & 4 & 1 & c & ... \\ \hline
		Related * & 5 & 2 & d & ... \\ \hline
		~ & 6 & 3 & e & ... \\ \hline
		Related * & 7 & 5 & f & ... \\ \hline
		~ & 8 & 6 & g & ... \\ \hline
		Selected \checkmark & 9 & 7 & h & ... \\ \hline
		~ & 10 & 7 & i & ... \\ \hline
	\end{tabular}
	\caption{Table Representation for the Semantic Related Group}
	\label{table:semanticrelatedTable}
\end{table}
  %%%%%%%%%%%%%%%%%%%%%%%%%%%%%%%%%%%%%%%%%%%%%%%%%%%%%%%%%%%%%%%%%%%%%%%%%%%%%
  %
%%%%%                      SECOND SECTION
 %%%
  %

\section{B}

BBB

  %%%%%%%%%%%%%%%%%%%%%%%%%%%%%%%%%%%%%%%%%%%%%%%%%%%%%%%%%%%%%%%%%%%%%%%%%%%%%
  %
%%%%%                         ANOTHER SECTION
 %%%
  %
\section{C}

CCC

