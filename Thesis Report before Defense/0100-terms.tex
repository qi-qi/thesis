  %%%%%%%%%%%%%%%%%%%%%%%%%%%%%%%%%%%%%%% -*- coding: utf-8; mode: latex -*- %%
  %
%%%%%                        T E R M I N O L O G I A
 %%%
  %

% $Id: 0100-terms.tex,v 1.2 2009/06/19 15:51:46 david Exp $
% $Log: 0100-terms.tex,v $
% Revision 1.2  2009/06/19 15:51:46  david
% *** empty log message ***
%
% Revision 1.1  2007/11/23 09:52:39  david
% *** empty log message ***
%
%

  %%%%%%%%%%%%%%%%%%%%%%%%%%%%%%%%%%%%%%%%%%%%%%%%%%%%%%%%%%%%%%%%%%%%%%%%%%%%%
  %
%%%%%
 %%%
  %

% a ordem não é relevante para o processamento, mas é-o para a gestão do
% conteúdo deste ficheiro.
% ATENÇÃO: maiúsculas e minúsculas são consideradas iguais.

  %%%%%%%%%%%%%%%%%%%%%%%%%%%%%%%%%%%%%%%%%%%%%%%%%%%%%%%%%%%%%%%%%%%%%%%%%%%%%
  %
%%%%%                     A  A  A  A  A  A  A  A  A
 %%%
  %

\def\AF{AF\index{AF}}

%--------------------------------------------------

\def\AFS{AFS\index{AFS}}
\nomenclature{AFS}{Andrew File System ou Advanced File
System~\cite{www:openafs,www:afscmu}. O Andrew File System é um sistema de
ficheiros distribuído. Foi inicialmente desenvolvido na Universidade de
Carnegie Mellon. O AFS apresenta várias vantagens sobre outros sistemas de
ficheiros, particularmente no que respeita às áreas de segurança e
escalabilidade.}

%--------------------------------------------------

\def\AlethGD{AlethGD\index{AlethGD}}

%--------------------------------------------------

\def\Amorfo{Amorfo\index{Amorfo}}
\index{analisador morfológico!Amorfo|see{Amorfo}}

%--------------------------------------------------

\def\API{API}

%--------------------------------------------------

\def\ArgoUML{ArgoUML\index{ArgoUML}}

%--------------------------------------------------

\def\ATA{ATA\index{ATA}}

%--------------------------------------------------

\def\AUTHOR{AUTHOR\index{AUTHOR}}
%\nomenclature{AUTHOR}{\AUTHOR: arquitecura de
%  geração de prosa narrativa~\cite{callaway01}. Ver a entrada
%  correspondente a \StoryBook.}

  %%%%%%%%%%%%%%%%%%%%%%%%%%%%%%%%%%%%%%%%%%%%%%%%%%%%%%%%%%%%%%%%%%%%%%%%%%%%%
  %
%%%%%                     B  B  B  B  B  B  B  B  B
 %%%
  %

\def\BLARK{BLARK\index{BLARK}}
%\nomenclature{BLARK}{Basic LAnguage Resource Kit.}

%--------------------------------------------------

\def\BRASILex{BRASILex\index{BRASILex}}
\nomenclature{BRASILex}{Léxico monolingue e multifuncional
  para a variante brasileira da Língua Portuguesa~\cite{wittmann00some}. A
  colecção compreende cerca de 65000 entradas (lemas) e os correspondentes
  1600 paradigmas de flexão. O conjunto de entradas inclui palavras
  compostas. Os paradigmas de flexão contêm informação relativa a clíticos
  e aos graus aumentativo e diminutivo. A informação morfológica apresenta
  uma fina granularidade e é conforme as recomendações \EAGLES{} (catálogo:
  ELRA-L0034 BrasiLEX Brazilian Portuguese lexicon).}

  %%%%%%%%%%%%%%%%%%%%%%%%%%%%%%%%%%%%%%%%%%%%%%%%%%%%%%%%%%%%%%%%%%%%%%%%%%%%%
  %
%%%%%                     C  C  C  C  C  C  C  C  C
 %%%
  %

  %%%%%%%%%%%%%%%%%%%%%%%%%%%%%%%%%%%%%%%%%%%%%%%%%%%%%%%%%%%%%%%%%%%%%%%%%%%%%
  %
%%%%%                     D  D  D  D  D  D  D  D  D
 %%%
  %

\def\DCR{DCR\index{DCR}}
\index{Data Category Registry|see{DCR}}
\nomenclature{DCR}{Data Category
Registry~\cite{lrec2004:wright04global,isotc37sc4:wright02data}. A \DCR{} é
um componente do Linguistic Annotation
Framework~\cite{isotc37sc4:ide02linguistic,ide03outline} que contém um
conjunto de categorias linguísticas definido
formalmente~\cite{lrec2004:ide04registry}.}

%--------------------------------------------------

\def\DCS{DCS\index{DCS}}
\index{Data Category Selection|see{DCS}}
\nomenclature{DCS}{Data Category Selection~\cite{isotc37sc4:dcr}:
subconjuntos de uma \DCR{} que reflectem vários domínios temáticos e várias
classes e funções de categorias de dados. A figura~\ref{fig:data:dcrdcs}
(página~\pageref{fig:data:dcrdcs}) apresenta a relação \DCR/\DCS.}

%--------------------------------------------------
% sec:content-determination
%
\nomenclature{Determinação de conteúdo}{Tarefa que decide que in\-for\-ma\-ção
  deve ser comunicada no documento de saída. Pode ser vista como o aspecto de
  conteúdo do planeador de documentos~\cite{reiter00building}
  (§\ref{sec:content-determination},
  página~§\pageref{sec:content-determination}). No contexto do projecto \RAGS,
  esta designação corresponde a toda a fase de planeamento do
  documento~(§\ref{sec:rags}, página~§\pageref{sec:rags}).}

%--------------------------------------------------

\def\docuplanner{DocuPlanner\index{arquitectura!DocuPlanner (sistema)}\index{modelos de geração!geração profunda!DocuPlanner (sistema)}\index{sistema!DocuPlanner}\index{document drafting!DocuPlanner (sistema)}\index{sistema!DocuPlanner}}
% \nomenclature{DocuPlanner}{\docuplanner{} -- Um sistema de preparação de rascunhos de documentos~\cite{branting99}.}

%--------------------------------------------------

\def\DTD{DTD\index{DTD}}
\def\DTL{DTL\index{DTL}}

%--------------------------------------------------

\def\DOM{DOM\index{DOM}}
\nomenclature{DOM}{Document Object Model~\cite{www:dom}. O Document Object
Model é uma interface neutra relativamente a plataformas ou linguagens
particulares. Esta interface permite acesso dinâmico ao conteúdo, estrutura
e estilo de documentos que sigam este padrão.}

  %%%%%%%%%%%%%%%%%%%%%%%%%%%%%%%%%%%%%%%%%%%%%%%%%%%%%%%%%%%%%%%%%%%%%%%%%%%%%
  %
%%%%%                     E  E  E  E  E  E  E  E  E
 %%%
  %

\def\EAGLES{EAGLES\index{EAGLES}}
\index{recomendações EAGLES|see{EAGLES}}

%--------------------------------------------------

\def\langpt{Português\index{língua!Português}}
\def\langes{Espanhol\index{língua!Espanhol}}
\def\langen{Inglês\index{língua!Inglês}}
\def\langfr{Francês\index{língua!Francês}}

%--------------------------------------------------

\def\Edite{Edite\index{Edite}}
\nomenclature{Edite}{Sistema desenvolvido para aceder em
  linguagem natural a uma base de dados de recursos turísticos dos quiosques
  multimédia do \INESC. O processo de acesso contempla três etapas: (a)
  análise morfológica -- através do \JSpell~\cite{almeida94jspell} --,
  responsável pela associação de informação morfo\pdash{}sintáctica às
  palavras da frase; (b) análise sintáctica (algoritmo de Earley), fase em que
  são geradas uma ou mais árvores sintácticas representantes da estrutura da
  frase; (c) análise semântica, onde é criada uma forma lógica que exprime o
  significado da frase em tratamento. O sistema é multilingue, suportando
  interacções em \langpt, \langes, \langen{} e \langfr~\cite{th:luisams}.}

%--------------------------------------------------

\index{entidade morfológica|see{unidade morfológica}}
\index{entidade sintáctica|see{unidade sintáctica}}
\index{entidade semântica|see{unidade semântica}}

%--------------------------------------------------

\def\EPLEXIC{EPLexIC\index{EPLexIC}}

%--------------------------------------------------

\def\EuroWordNet{EuroWordNet\index{EuroWordNet}}
% \nomenclature{EuroWordNet}{\EuroWordNet{}~\cite{www:eurowordnet}.}

%--------------------------------------------------

\def\EUROTRA{EUROTRA\index{EUROTRA}}

  %%%%%%%%%%%%%%%%%%%%%%%%%%%%%%%%%%%%%%%%%%%%%%%%%%%%%%%%%%%%%%%%%%%%%%%%%%%%%
  %
%%%%%                     F  F  F  F  F  F  F  F  F
 %%%
  %

%--------------------------------------------------

\def\FrameNet{FrameNet\index{FrameNet}}

%--------------------------------------------------



%--------------------------------------------------

\def\FUF{FUF\index{FUF}}
\nomenclature{FUF}{Iniciais de \textsl{Funcional Unification
    For\-mal\-ism}~\cite{elhadad93a,elhadad93b}. Formalismo de
  unificação baseado no proposto originalmente por~\citeA{kay79}.}

%--------------------------------------------------

\nomenclature{Função de correspondência semântica}{No contexto da arquitectura
  dos dados (capítulo~\ref{ch:smartglue}, página~\pageref{ch:smartglue}),
  função que traduz a semântica dos dados que fluem através de uma ligação
  entre dois módulos em comunicação. Ver~§\ref{sec:semantica}
  (página~\pageref{sec:semantica}), e definição~\ref{def:semantics}.}

  %%%%%%%%%%%%%%%%%%%%%%%%%%%%%%%%%%%%%%%%%%%%%%%%%%%%%%%%%%%%%%%%%%%%%%%%%%%%%
  %
%%%%%                     G  G  G  G  G  G  G  G  G
 %%%
  %

\def\Galaxy{Galaxy\index{Galaxy Communicator}}

%--------------------------------------------------

\def\Galinha{Galinha\index{Galinha}}
\def\GATE{GATE\index{GATE}}

%--------------------------------------------------

\def\Genelex{Genelex\index{Genelex}}

%--------------------------------------------------

\def\GGG{Galinha Galaxy Gateway\index{Galinha!Galaxy Gateway}}

%--------------------------------------------------

\def\GLOSIX{GLOSIX\index{GLOSIX}}
\index{Multext!GLOSIX|see{GLOSIX}}
\index{General Lingware Open System Environment|see{GLOSIX}}

  %%%%%%%%%%%%%%%%%%%%%%%%%%%%%%%%%%%%%%%%%%%%%%%%%%%%%%%%%%%%%%%%%%%%%%%%%%%%%
  %
%%%%%                     H  H  H  H  H  H  H  H  H
 %%%
  %

%--------------------------------------------------

\def\HTML{HTML\index{HTML}}

%--------------------------------------------------

\index{arquitectura!HYLITE+ (sistema)}
\index{sistema!HYLITE+}
\index{tempo real|see{sistemas de tempo real}}
\index{sistemas de tempo real!HYLITE+}

  %%%%%%%%%%%%%%%%%%%%%%%%%%%%%%%%%%%%%%%%%%%%%%%%%%%%%%%%%%%%%%%%%%%%%%%%%%%%%
  %
%%%%%                     I  I  I  I  I  I  I  I  I
 %%%
  %

\def\IMDI{IMDI\index{IMDI}}

%--------------------------------------------------

\def\ILEX{ILEX\index{ILEX}}

%--------------------------------------------------

\def\INTERA{INTERA\index{INTERA}}
%\index{projecto!INTERA|see{INTERA}}

%--------------------------------------------------

\def\INESC{INESC\index{INESC}}

%--------------------------------------------------

\def\ISLE{ISLE\index{ISLE}}
%\index{projecto!ISLE|see{ISLE}}

%--------------------------------------------------

\def\ISO{ISO\index{ISO}}
\index{ISO!TC37|see{TC37}}

%--------------------------------------------------

\def\ispell{ispell\index{ispell}}
\index{analisador morfológico!ispell|see{ispell}}
\index{international ispell|see{ispell}}
\nomenclature{ispell}{International Ispell é um programa interactivo para
  verificação ortográfica que suporta várias línguas Europeias~\cite{www:ispell}.
}

  %%%%%%%%%%%%%%%%%%%%%%%%%%%%%%%%%%%%%%%%%%%%%%%%%%%%%%%%%%%%%%%%%%%%%%%%%%%%%
  %
%%%%%                     J  J  J  J  J  J  J  J  J
 %%%
  %

\def\Java{Java\index{Java}}
\def\JavaScript{JavaScript\index{JavaScript}}

%--------------------------------------------------

\def\JDBC{JDBC\index{JDBC}}
\index{Java Database Connectivity|see{JDBC}}

%--------------------------------------------------

\def\JSpell{JSpell\index{JSpell}}
\index{analisador morfológico!JSpell|see{JSpell}}

  %%%%%%%%%%%%%%%%%%%%%%%%%%%%%%%%%%%%%%%%%%%%%%%%%%%%%%%%%%%%%%%%%%%%%%%%%%%%%
  %
%%%%%                     K  K  K  K  K  K  K  K  K
 %%%
  %

\def\Kerberos{Kerberos\index{Kerberos}}
\nomenclature{Kerberos}{Protocolo de autenticação em
rede~\cite{steiner88kerberos,neuman94kerberos}. Está desenhado para
providenciar autenticação forte entre aplicações cliente/servidor através de
criptografia de chave secreta.}

  %%%%%%%%%%%%%%%%%%%%%%%%%%%%%%%%%%%%%%%%%%%%%%%%%%%%%%%%%%%%%%%%%%%%%%%%%%%%%
  %
%%%%%                     L  L  L  L  L  L  L  L  L
 %%%
  %
\def\LDAP{LDAP\index{LDAP}}
\index{Lightweight Directory Access Protocol|see{LDAP}}
\nomenclature{LDAP}{Lightweight Directory Access
Protocol~\cite{www:openldap} é um conjunto de protocolos para acesso a
informação organizada em directórios. O \LDAP{} baseia-se na norma
X.500~\cite{iso:iec:9594:1}, sendo, no entanto, mais simples e
interoperável com protocolos Internet.}

%--------------------------------------------------

\def\LDC{LDC\index{LDC}}
\index{Linguistic Data Consortium|see{LDC}}

%--------------------------------------------------

\def\LISP{LISP\index{LISP}}

%--------------------------------------------------

\def\LUSOlex{LUSOlex\index{LUSOlex}}
\nomenclature{LUSOlex}{Léxico monolingue e multifuncional
  para a variante europeia da Língua Portuguesa~\cite{wittmann00some}. A
  colecção compreende cerca de 61000 entradas (lemas) e os correspondentes
  1600 paradigmas de flexão. O conjunto de entradas inclui palavras
  compostas. Os paradigmas de flexão contêm informação relativa a clíticos e
  aos graus aumentativo e diminutivo. A informação morfológica apresenta uma
  fina granularidade e é conforme as recomendações \EAGLES{} (catálogo:
  ELRA-L0033 LUSOlex European Portuguese Lexicon).}

  %%%%%%%%%%%%%%%%%%%%%%%%%%%%%%%%%%%%%%%%%%%%%%%%%%%%%%%%%%%%%%%%%%%%%%%%%%%%%
  %
%%%%%                     M  M  M  M  M  M  M  M  M
 %%%
  %

%--------------------------------------------------

\def\m4{GNU \texttt{m4}\index{m4}}
\index{GNU m4|see{m4}}
\nomenclature{m4}{Processador de macros. \m4{} possui funções internas para inclusão de ficheiros,
  execução de comandos, aritmética, etc.}

%--------------------------------------------------

\def\marv{MARv\index{MARv}}

%--------------------------------------------------

\def\MILE{MILE\index{MILE}}
\index{Multilingual ISLE Lexical Entry|see{MILE}}

%--------------------------------------------------

\def\ModelExplainer{ModelExplainer\index{ModelExplainer}}

%--------------------------------------------------

\def\monge{Monge\index{monge}}
\index{gerador morfológico|see{monge}}
\index{realização de superfície!morfológica|see{monge}}

%--------------------------------------------------

\def\Multext{Multext\index{Multext}}
\index{projecto!Multext|see{Multext}}

%--------------------------------------------------

\def\Multilex{Multilex\index{Multilex}}

%--------------------------------------------------

\def\MySQL{MySQL\index{MySQL}}

  %%%%%%%%%%%%%%%%%%%%%%%%%%%%%%%%%%%%%%%%%%%%%%%%%%%%%%%%%%%%%%%%%%%%%%%%%%%%%
  %
%%%%%                     N  N  N  N  N  N  N  N  N
 %%%
  %
\def\NEMLAR{NEMLAR\index{NEMLAR}}

%--------------------------------------------------

\def\nlpfarm{NLPFARM\index{NLPFARM}}
\def\NOMLEX{NOMLEX\index{NOMLEX}}

%--------------------------------------------------

\def\noweb{\texttt{noweb}\index{noweb}}
\index{programação literária!noweb|see{noweb}}
\nomenclature{noweb}{Ferramenta de programação
  literária independente da linguagem de pro\-gra\-ma\-ção~\cite{www:noweb}.}

  %%%%%%%%%%%%%%%%%%%%%%%%%%%%%%%%%%%%%%%%%%%%%%%%%%%%%%%%%%%%%%%%%%%%%%%%%%%%%
  %
%%%%%                     O  O  O  O  O  O  O  O  O
 %%%
  %

\def\ODBC{ODBC\index{ODBC}}
\index{Open Database Connectivity|see{ODBC}}

%--------------------------------------------------

\def\OWL{OWL\index{OWL}}
\nomenclature{OWL}{Web Ontology Language~\cite{www:owl}. \OWL{} é uma
linguagem que permite a definição de ontologias baseadas na Web para permitir
a integração de dados e a interoperabilidade entre comunidades. \OWL{} parte
de \RDF{} e \RDFS{} e adiciona vocabulário para a descrição de propriedades e
classes: relações entre classes, cardinalidade, igualdade, entre
outros~\cite{www:owlfeatures,w3c:sws-pressrelease-040210}. \OWL{} permite a
definição de ontologias compatíveis com a arquitectura da Web, em geral, e com
a Semantic Web, em particular.}

  %%%%%%%%%%%%%%%%%%%%%%%%%%%%%%%%%%%%%%%%%%%%%%%%%%%%%%%%%%%%%%%%%%%%%%%%%%%%%
  %
%%%%%                     P  P  P  P  P  P  P  P  P
 %%%
  %

\index{paradigma de flexão!fonético|see{forma fonética}}
\index{paradigma de flexão!gráfico|see{forma gráfica}}

%--------------------------------------------------

\def\PAROLE{PA\-RO\-LE\index{PAROLE}}
\index{projecto!PAROLE|see{PAROLE}}
\index{LE-PAROLE|see{PAROLE}}

%--------------------------------------------------

\def\Palavroso{Palavroso\index{Palavroso}}
\def\pasmo{PAsMo\index{PAsMo}}
\def\PEAR{PEAR\index{PEAR}}

%--------------------------------------------------

\def\PEBA{PEBA\index{PEBA-II}}
\index{sistema!PEBA-II|see{PEBA-II}}
\index{modos de interacção!monólogo!PEBA-II (sistema)|see{PEBA-II}}
\index{modelos de geração!geração profunda!PEBA-II (sistema)|see{PEBA-II}}
% \nomenclature{PEBA-II}{\PEBA-II -- ~\cite{dale96}.}

%--------------------------------------------------

\index{perfil do utilizador|see{modelo do utilizador}}
\index{preferências do utilizador|see{modelo do utilizador}}

%--------------------------------------------------

\def\PHP{PHP\index{PHP}}
\def\Poseidon{Poseidon\index{Poseidon}}

%--------------------------------------------------

\def\plandoc{PLANDoc\index{PLANDoc}}
\index{sistema!PLANDoc|see{PLANDoc}}
\index{modelos de geração!geração profunda!PLANDoc (sistema)|see{PLANDoc}}
% \nomenclature{PLANDoc}{\plandoc{} -- ~\cite{mckeown94}.}

%--------------------------------------------------

\nomenclature{Planeador de frases}{Designação alternativa para o
micro\pdash{}planeador (projecto \RAGS).}
\index{planeador de texto|see{micro\pdash{}planeador}}
\nomenclature{Planeador de texto}{Designação alternativa para o micro\pdash{}planeador.}

%--------------------------------------------------

\def\POWER{POWER\index{POWER}}
\index{sistema!POWER|see{POWER}}
\index{modos de interacção!monólogo!POWER (sistema)|see{POWER}}
\index{modelos de geração!geração profunda!POWER (sistema)|see{POWER}}

%--------------------------------------------------

\def\Python{Python\index{Python}}

  %%%%%%%%%%%%%%%%%%%%%%%%%%%%%%%%%%%%%%%%%%%%%%%%%%%%%%%%%%%%%%%%%%%%%%%%%%%%%
  %
%%%%%                     R  R  R  R  R  R  R  R  R
 %%%
  %

\def\RAGS{RAGS\index{RAGS}}
%--------------------------------------------------

\def\RDF{RDF\index{RDF}}
\nomenclature{RDF}{Resource Definition Framework~\cite{www:rdf}. \RDF{} é
parte da W3C Metadata Activity (\url{http://www.w3.org/Metadata/}). O
objectivo desta actividade, e do \RDF{} em particular, é a produção de uma
linguagem para o intercâmbio de descrições dos recursos da Web. As descrições
destinam-se a usos
automáticos~\cite{www:rdf,www:rdfxml,w3c:sws-pressrelease-040210}.}

\def\RDFS{RDFS\index{RDFS}}
\index{RDF Schema|see{RDFS}}
\nomenclature{RDFS}{\RDF{} Schema~\cite{www:rdfs}. \RDFS{} é uma extensão
semântica do \RDF{}, providenciando mecanismos que permitem a descrição de
grupos de recursos relacionados, bem como as relações entre esses recursos. As
descrições são escritas de acordo com \RDF{}. Os recursos são utilizadas para
determinar as características de outros recursos, tais como domínios e gamas
de propriedades.}

  %%%%%%%%%%%%%%%%%%%%%%%%%%%%%%%%%%%%%%%%%%%%%%%%%%%%%%%%%%%%%%%%%%%%%%%%%%%%%
  %
%%%%%                     S  S  S  S  S  S  S  S  S
 %%%
  %

\def\SGML{SGML\index{SGML}}
\index{ISO!8879|see{SGML}}

%--------------------------------------------------

\def\SIMPLE{SIM\-PLE\index{SIMPLE}}
\index{projecto!SIMPLE|see{SIMPLE}}

%--------------------------------------------------

\def\SMorph{SMorph\index{SMorph}}
\index{analisador morfológico!SMorph|see{SMorph}}

%--------------------------------------------------

\def\SOAP{SOAP\index{SOAP}}

%--------------------------------------------------

\def\SPEECHDAT{SPEECHDAT\index{SPEECHDAT}}

%--------------------------------------------------

\def\SQL{SQL\index{SQL}}

%--------------------------------------------------

\def\StoryBook{StoryBook\index{StoryBook}}
\nomenclature{StoryBook}{\StoryBook{} é uma implementação da arquitecura de
  geração de prosa narrativa AUTHOR~\cite{callaway00narrative}. O sistema
  executa as funções de planeamento da narrativa, assim como as funções de
  geração de língua natural. O texto final é construído utilizando o
  realizador de superfície \FUF/\SURGE~\cite{elhadad96}. As histórias geradas
  situam-se no domínio do Capuchinho Vermelho~\index{Capuchinho
    Vermelho}\index{Little Red Riding Hood|see{Capuchinho Vermelho}}.}

%--------------------------------------------------

\def\SURGE{SURGE\index{SURGE}}
\index{arquitectura modular!realizador de superfície!SURGE (sistema)|see{SURGE}}
\index{sistema!SURGE|see{SURGE}}
\nomenclature{SURGE}{Realizador de superfície para Inglês (Systemic
  Unification Realization Grammar of English). Uma apresentação do sistema é
  feita em~\citeA{elhadad96}.}

%--------------------------------------------------

\def\susana{SuSAna\index{SuSAna}}

  %%%%%%%%%%%%%%%%%%%%%%%%%%%%%%%%%%%%%%%%%%%%%%%%%%%%%%%%%%%%%%%%%%%%%%%%%%%%%
  %
%%%%%                     T  T  T  T  T  T  T  T  T
 %%%
  %

\def\TEI{TEI\index{TEI}}
\index{Text Encoding Initiative|see{TEI}}

%--------------------------------------------------

\def\TIPSTER{TIPSTER\index{TIPSTER}}

%--------------------------------------------------

\def\TCxxxvii{TC37\index{TC37}}
\def\TCxxxviiSCiv{TC37/SC4\index{TC37!SC4}}

%--------------------------------------------------

\def\TRIPS{TRIPS\index{TRIPS}\index{modos de interacção!diálogo!TRIPS (sistema)}\index{sistema!TRIPS}}

  %%%%%%%%%%%%%%%%%%%%%%%%%%%%%%%%%%%%%%%%%%%%%%%%%%%%%%%%%%%%%%%%%%%%%%%%%%%%%
  %
%%%%%                     U  U  U  U  U  U  U  U  U
 %%%
  %

\def\UML{UML\index{UML}}

%--------------------------------------------------

\def\Unix{Unix\index{Unix}}

  %%%%%%%%%%%%%%%%%%%%%%%%%%%%%%%%%%%%%%%%%%%%%%%%%%%%%%%%%%%%%%%%%%%%%%%%%%%%%
  %
%%%%%                     U  U  U  U  U  U  U  U  U
 %%%
  %

\def\vnACCMS{vnACCMS\index{vnACCMS}}
\nomenclature{vnACCMS}{Sistema que realiza tarefas de segmentação
  de palavras e etiquetação morfológica. O sistema utiliza o formato de
  representação para recursos linguísticos tal como definido no âmbito do
  trabalho da equipa \ISO{} \TCxxxviiSCiv{}. Ver
  \url{http://www.loria.fr/equipes/led/outils.php}~\cite{lrec2004:nguyen04developping}.}

  %%%%%%%%%%%%%%%%%%%%%%%%%%%%%%%%%%%%%%%%%%%%%%%%%%%%%%%%%%%%%%%%%%%%%%%%%%%%%
  %
%%%%%                     W  W  W  W  W  W  W  W  W
 %%%
  %

\def\WeatherReporter{WeatherReporter\index{WeatherReporter}}

%--------------------------------------------------

\def\WordNet{WordNet\index{WordNet}}
\nomenclature{WordNet}{Léxico semântico para Inglês~\cite{www:wordnet,fellbaum98wordnet}.}

%--------------------------------------------------

\def\WSDL{WSDL\index{WSDL}}

%--------------------------------------------------

\def\WSEL{WSEL\index{WSEL}}
\def\WSFL{WSFL\index{WSFL}}

  %%%%%%%%%%%%%%%%%%%%%%%%%%%%%%%%%%%%%%%%%%%%%%%%%%%%%%%%%%%%%%%%%%%%%%%%%%%%%
  %
%%%%%                     X  X  X  X  X  X  X  X  X
 %%%
  %

\def\XA{XA\index{XA}}
\index{analisador morfológico!XA|see{XA}}

%--------------------------------------------------

\def\XMI{XMI\index{XMI}}
\nomenclature{XMI}{\XML{} Metadata Interchange~\cite{www:xmi}. \XMI{} é um
enquadramento para a definição, intercâmbio, manipulação e integração de
objectos \XML. As normas baseadas em \XMI{} permitem a integração de
ferramentas e repositórios~\cite{www:xmi}.}

%--------------------------------------------------

\def\XML{XML\index{XML}}
\nomenclature{XML}{Extensible Markup Language~\cite{www:xml} é um formato de texto, simples e
flexível, derivado de \SGML{} (\ISO{}~8879)~\cite{iso:8879}.}

%--------------------------------------------------

\def\XSD{XSD\index{XSD}}
\nomenclature{XSD}{\XML{} Schema Definition~\cite{www:xsd}. Os esquemas \XML{}
expressam vocabulários partilhados e providenciam formas de definir a
estrutura, conteúdo e semântica de documentos \XML. Ver
\url{www.oasis-open.org/cover/schemas.html}.}

%--------------------------------------------------

\def\XSL{XSL\index{XSL}}
\def\XSLT{XSLT\index{XSLT}}
\nomenclature{XSLT}{\XSL{} Transformations~\cite{www:xslt} é uma linguagem para
  transformar documentos \XML. A transformação \XSLT{} descreve as regras para
  transformar uma árvore de entrada numa árvore de saída independente da
  árvore original. A linguagem permite filtrar a árvore original assim como a
  adição de estruturas arbitrárias.}

  %
 %%%
%%%%%                                F   I   M
  %
  %%%%%%%%%%%%%%%%%%%%%%%%%%%%%%%%%%%%%%%%%%%%%%%%%%%%%%%%%%%%%%%%%%%%%%%%%%%%%
