  %%%%%%%%%%%%%%%%%%%%%%%%%%%%%%%%%%%%%%% -*- coding: utf-8; mode: latex -*- %%
  %
%%%%%                         CHAPTER
 %%%
  %

% $Id: 1020-lorem-ipsum.tex,v 1.2 2009/06/19 15:51:46 david Exp $
% $Log: 1020-lorem-ipsum.tex,v $
% Revision 1.2  2009/06/19 15:51:46  david
% *** empty log message ***
%
% Revision 1.1  2007/11/23 09:52:39  david
% *** empty log message ***
%
%

  %%%%%%%%%%%%%%%%%%%%%%%%%%%%%%%%%%%%%%%%%%%%%%%%%%%%%%%%%%%%%%%%%%%%%%%%%%%%%
  %
%%%%%                           HEAD MATTER
 %%%
  %

\chapter{Introduction}
%\addcontentsline{lof}{chapter}{\thechapter\quad Lorem Ipsum}
%\addcontentsline{lot}{chapter}{\thechapter\quad Lorem Ipsum}
\label{ch:Introduction}

%\begin{quotation}
%  {\small\it Neque porro quisquam est qui dolorem ipsum quia dolor sit amet, consectetur, adipisci velit...}
%
%{\small\it -- Cerico}
%\end{quotation}



  %%%%%%%%%%%%%%%%%%%%%%%%%%%%%%%%%%%%%%%%%%%%%%%%%%%%%%%%%%%%%%%%%%%%%%%%%%%%%
  %
%%%%%                        FIRST SECTION
 %%%
  %

\section{Motivation}

\subsection{The De Facto Industrial Standard in Big Data Era}

The \textit{Apache Hadoop}~\cite{apachehadoop} ecosystem has become the de facto industrial standard to store, process and analyze large data sets in the big data era~\cite{cloudera}. It is widely used as a computational platform for a variety of areas including search engines, data warehousing, behavioral analysis, natural language processing, genomic analysis, image processing, etc~\cite{shvachko2011apache}. 

\noindent The \textit{Hadoop Distributed File System} (HDFS) is the storage layer for Apache Hadoop, which enables petabytes of data to be persisted on clusters of commodity hardware at relatively low cost~\cite{borthakur2008hdfs}. Inspired by the \textit{Google File System} (GFS)~\cite{ghemawat2003google}, the namespace, \textit{metadata}, is decoupled from data and stored in-memory on a single server, called the \textit{NameNode}. The file datasets are stored as sequences of blocks and replicated across potentially thousands of machines for fault tolerance.

\subsection{Limitation in HDFS}

Built upon the single namespace server, \textit{the NameNode}, architecture, one well-known limitation of HDFS is the limitation to growth~\cite{shvachko2010hdfs}. Since the metadata is kept in-memory for fast operation in NameNode, the number of file objects in the filesystem is limited by the amount of memory of the NameNode. 

\noindent Approximately, the size of the metadata for a single file object having two blocks (replicated three times by default) is 600 bytes. As a rule of thumb, for one petabyte physical storage, it requires one gigabyte metadata in memory~\cite{shvachko2010hdfs}. Table~\ref{table:memoryRequirement} gives an estimation of the memory requirement and its related physical storage capacity for different number of files.

\begin{table}[h]
	\centering
	\begin{tabular}{|c|c|c|}
		\hline
		\textbf{Number of Files} & \textbf{Memory Requirement} & \textbf{Physical Storage} \\ \hline
		1 million       & 0.6 GB             & 0.6 PB           \\ \hline
		100 million     & 60 GB              & 60 PB            \\ \hline
		1 billion       & 600 GB             & 600 PB           \\ \hline
		2 billion       & 1200 GB            & 1200 PB          \\ \hline
	\end{tabular}
	\caption{Memory Requirement for Related Storage Capacity in HDFS}
	\label{table:memoryRequirement}
\end{table}

  %%%%%%%%%%%%%%%%%%%%%%%%%%%%%%%%%%%%%%%%%%%%%%%%%%%%%%%%%%%%%%%%%%%%%%%%%%%%%
  %
%%%%%                      SECOND SECTION
 %%%
  %

\section{Problem Statement}

BBB

  %%%%%%%%%%%%%%%%%%%%%%%%%%%%%%%%%%%%%%%%%%%%%%%%%%%%%%%%%%%%%%%%%%%%%%%%%%%%%
  %
%%%%%                         ANOTHER SECTION
 %%%
  %
\section{Contribution}

CCC

  %%%%%%%%%%%%%%%%%%%%%%%%%%%%%%%%%%%%%%%%%%%%%%%%%%%%%%%%%%%%%%%%%%%%%%%%%%%%%
  %
%%%%%                          LAST SECTION
 %%%
  %

\section{Document Structure}

[To be Added]

  %
 %%%
%%%%%                        THE END
  %
  %%%%%%%%%%%%%%%%%%%%%%%%%%%%%%%%%%%%%%%%%%%%%%%%%%%%%%%%%%%%%%%%%%%%%%%%%%%%%

%%% Local Variables: 
%%% mode: latex
%%% TeX-master: "tese"
%%% End: 
